\begin{titlepage}
\vspace*{-1cm}
\newlength{\links}
\setlength{\links}{0.9cm}
\setlength{\TPHorizModule}{1cm}
\setlength{\TPVertModule}{1cm}
%\textblockorigin{0pt}{0pt}

\sf
\LARGE

\begin{textblock}{16.5}(2.8,2.7)
 \hspace*{-0.8cm} \textbf{University of Duisburg-Essen} \\
 \hspace*{-1.15cm} \rule{5mm}{5mm} \hspace*{0.0cm} Faculty of Engineering\\
 \large{}Department of Computer and Cognitive Sciences\\
\end{textblock}

%Hier Titel, Name, und Matrikelnummer eintragen, \\ make a newline
\begin{textblock}{14.5}(3.2,7.5)
\begin{center}
  \large
{\bf Doctoral Dissertation} \\[1cm]
{ \LARGE  \bf Large data visualization} \\[1.3cm]
Thomas Fogal\\
Matriculation Number: 300306200
\end{center}
\end{textblock}

\begin{textblock}{10}(10.5,15.5)
\includegraphics[width=.94\textwidth]{images/unilogo}\\
\normalsize
\raggedleft
Department of Computer and Cognitive Sciences \\
Faculty of Engineering \\
University of Duisburg-Essen \\[2ex]

\today\\[13ex]
%February 28, 2015\\[13ex]
\raggedright
% Supervisors
{\bf Supervisor:} \\
Prof. Dr. rer. nat. Jens Kr\"uger\\

{\bf Reviewers:}\\
Prof. Dr. rer. nat. Jens Kr\"uger\\
Prof. Chris Johnson\\
\todo{Prof. Dr. J\"urgen Ziegler ??}\\
\todo{????}
\end{textblock}

\end{titlepage}

%
% additional declaration
%

\clearpage
\thispagestyle{empty}
~
% \vfill
\begin{flushleft}
  \textbf{Eidesstattliche Versicherung / Statement in lieu of an oath:}\\
  Ich versichere hiermit an Eides Statt, dass ich die vorliegende
  Arbeit selbstst\"andig verfasst und keine anderen als die angegebenen
  Quellen und Hilfsmittel verwendet habe.\\

  I hereby confirm that I have written this dissertation on my own
  and that I have not used any media or materials other than the ones
  referred to in this dissertation.\\[\baselineskip]

	Duisburg, February 28, 2015\\

% \vspace{4cm}
% 	\textbf{Einverst\"andniserkl\"arung / Declaration of Consent:}\\
% 	Ich bin damit einverstanden, dass meine (bestandene) Arbeit in beiden Versionen in die Bibliothek der
% Informatik aufgenommen und damit ver\"offentlicht wird.\\
% 	I agree to make both versions of my thesis (with a passing grade) accessible to the public by having
% them added to the library of the Computer Science Department.\\[\baselineskip]
% 	Duisburg, August 01, 2014
% \vspace{3cm}
\end{flushleft}

\clearpage

\section*{Zusammenfassung / Abstract}

\todo{auf Deutsch: lorem ipsum dolor sit amet ...}

\vspace{1em}

\hrule{}
\vspace{1em}

Visualization has emerged as a critical component in deriving
understanding from the vast amounts of data generated from both
simulations and modern scanning technologies such as computed
tomography.  The organization of these data dictates how they are
algorithmically processed and thereby the performance of processes
that operate on the data.  For these performance reasons as well as
simplicity of implementation, a regular $N$D grid organization has
heretofore dominated in the simulation, medical, and visualization
domains.

Yet the regular organization of data alone is not enough.  The pace of
data growth has exceeded that of hardware growth for many years now,
and the ensuing performance gap creates difficulties for visualization
algorithms.  As basically all sciences move to a data-centric approach,
these performance limitations become the limiting factor in forward
scientific progress.

To deal with this delude of data, many have turned to \textit{in situ}
visualization: coupling simulation and visualization software together
in an effort to minimize delay.  This is presently a daunting process,
one that cannot be sustained at large scale with the dearth of software
engineering resources across the research community.

This dissertation presents a number of community-vetted ideas aimed at
removing these barriers.  The algorithm targetted is volume rendering,
which is commonly a first step on the way to data understanding for
a wide variety of scientific disciplines.  As we dissipate these
challenges, we turn to the related problem of integrating volume
rendering solutions with
simulation software \textit{in situ}, specifically focusing on ways to
minimize the engineering investment.
