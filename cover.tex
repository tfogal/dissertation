\begin{titlepage}
\vspace*{-1cm}
\newlength{\links}
\setlength{\links}{0.9cm}
\setlength{\TPHorizModule}{1cm}
\setlength{\TPVertModule}{1cm}
%\textblockorigin{0pt}{0pt}

\sf
\LARGE

\begin{textblock}{16.5}(2.8,2.7)
 \hspace*{-0.8cm} \textbf{University of Duisburg-Essen} \\
 \hspace*{-1.15cm} \rule{5mm}{5mm} \hspace*{0.0cm} Faculty of Engineering\\
 \large{}Department of Computer and Cognitive Sciences\\
\end{textblock}

%Hier Titel, Name, und Matrikelnummer eintragen, \\ make a newline
\begin{textblock}{14.5}(3.2,7.5)
\begin{center}
  \large
{\bf Doctoral Dissertation} \\[1cm]
{ \LARGE  \bf Visualizing and understanding large regular data} \\[1.3cm]
Thomas Fogal\\
Matriculation Number: 300306200
\end{center}
\end{textblock}

\begin{textblock}{10}(10.5,15.5)
\includegraphics[width=.94\textwidth]{images/unilogo}\\
\normalsize
\raggedleft
Department of Computer and Cognitive Sciences \\
Faculty of Engineering \\
University of Duisburg-Essen \\[2ex]

\today\\[13ex]
%February 28, 2015\\[13ex]
\raggedright
% Supervisors
{\bf Supervisor:} \\
Prof. Dr. rer. nat. Jens Kr\"uger\\

{\bf Reviewers:}\\
Prof. Dr. rer. nat. Jens Kr\"uger\\
Prof. Chris Johnson\\
%\todo{Prof. Dr. J\"urgen Ziegler ??}\\
%\todo{????}
\end{textblock}

\end{titlepage}

%
% additional declaration
%

\clearpage
\thispagestyle{empty}
~
% \vfill
\begin{flushleft}
  \textbf{Eidesstattliche Versicherung / Statement in lieu of an oath:}\\
  Ich versichere hiermit an Eides Statt, dass ich die vorliegende
  Arbeit selbstst\"andig verfasst und keine anderen als die angegebenen
  Quellen und Hilfsmittel verwendet habe.\\

  I hereby confirm that I have written this dissertation on my own
  and that I have not used any media or materials other than the ones
  referred to in this dissertation.\\[\baselineskip]

	Santa Clara, CA, USA\\
	\today{}\\%February 28, 2015\\

% \vspace{4cm}
% 	\textbf{Einverst\"andniserkl\"arung / Declaration of Consent:}\\
% 	Ich bin damit einverstanden, dass meine (bestandene) Arbeit in beiden Versionen in die Bibliothek der
% Informatik aufgenommen und damit ver\"offentlicht wird.\\
% 	I agree to make both versions of my thesis (with a passing grade) accessible to the public by having
% them added to the library of the Computer Science Department.\\[\baselineskip]
% 	Duisburg, August 01, 2014
% \vspace{3cm}
\end{flushleft}

\clearpage

\section*{Zusammenfassung / Abstract}

Die Visualisierung ist ein wesentlicher Bestandteil, wenn es um das
Verstehen enorm gro\ss{}er Datenmengen geht, die sowohl in Simulationen
als auch durch bildgebende Verfahren wie die Computertomographie
entstehen k\"onnen. Die Anordnung dieser Daten bestimmt hierbei,
wie diese algorithmisch verarbeitet werden k\"onnen, und hat somit
Einfluss auf die Effizienz derjeniger Prozesse, die auf solchen Daten
operieren. Aus diesen Performanzgr\"unden und auf Grund der Einfachheit
der Implementierung hat die regul\"are $N$D Gitteranordnung die
Bereiche der Simulation, Medizin und Visualisierung dominiert.

Allerdings reicht die regul\"are Anordnung der Daten nicht aus. Die
Geschwindigkeit, mit der die Datenmengen wachsen, \"ubersteigt
den Hardwarewachstum seit vielen Jahren, und die so entstandene
Leistungsabstand sorgt f\"ur Schwierigkeiten im Bereich der
Visualisierungsalgorithmen. Da sich beinahe alle Wissenschaften in
die Richtung von datenzentrierten Verfahren bewegen, stellen diese
Leistungseinschr\"ankungen einen limitierenden Faktor f\"ur den
wissenschaftlichen Fortschritt dar.

Um diese Datenmengen zu bew\"altigen, wird von vielen die sogenannte
\textit{in-situ}-Visualisierung eingesetzt. Dabei werden die
Simulation und die Visualisierung verbunden, um Verz\"ogerungen zu
minimieren. Momentan ist dies ein m\"uhseliger Prozess, welcher auf
Grund von mangelnden
Software-Engineering-Ressourcen nicht im gr\"o\ss{}eren Ausma\ss{} im
akademischen Umfeld durchf\"uhrbar ist.

Diese Dissertation demonstriert einige, durch die Community bereits
\"uberpr\"ufte Ideen, um die genannten H\"urden zu eliminieren. Der
Algorithmus im Fokus ist dabei Volumengrafik, eine weit verbreitete
Methode f\"ur das Datenverst\"andnis in vielen wissenschaftlichen
Disziplinen. W\"ahrend wir an der Problemstellung arbeiten, kombinieren
wir L\"osungen f\"ur Volumengrafik mit
Simulationssoftware auf \textit{in-situ}-Basis und schlagen dabei
verschiedene Wege ein, um besonders den Engineering-Aufwand zu
minimieren.


\vspace{1em}

\hrule{}
\vspace{1em}

Visualization has emerged as a critical component in deriving
understanding from the vast amounts of data generated from both
simulations and modern scanning technologies such as computed
tomography.  The organization of these data dictates how they are
algorithmically processed and thereby the performance of processes
that operate on the data.  For these performance reasons as well as
simplicity of implementation, a regular $N$D grid organization has
heretofore dominated in the simulation, medical, and visualization
domains.

Yet the regular organization of data alone is not enough.  The pace of
data growth has exceeded that of hardware growth for many years now,
and the ensuing performance gap creates difficulties for visualization
algorithms.  As basically all sciences move to a data-centric approach,
these performance limitations become the limiting factor in forward
scientific progress.

To deal with this delude of data, many have turned to \textit{in situ}
visualization: coupling simulation and visualization software together
in an effort to minimize delay.  This is presently a daunting process,
one that cannot be sustained at large scale with the dearth of software
engineering resources across the research community.

This dissertation presents a number of community-vetted ideas aimed at
removing these barriers.  The algorithm targeted is volume rendering,
a popular method for data understanding in a number of scientific
disciplines.  As we dissipate these challenges, we turn to the related
problem of integrating volume rendering solutions with
simulation software \textit{in situ}, specifically focusing on ways to
minimize the engineering investment.
