\section*{Acknowledgements}

Certainly, the primary acknowledgement should go to my advisor and
friend through this whole endeavor, Jens Kr\"uger.  It would be
impossible to overstate his contribution to this work, from developing
arguments to coding to writing and presenting; you would surely not be
reading this work today, if not for Jens.

A special thanks is also due to Hank Childs.  In addition to
significant contributions in writing the paper associated with
Chapter~\ref{chp:multiscale} and help integrating code with VisIt,
Hank's constant encouragement and praise were often exactly what I
needed to keep pushing after setbacks.

I was fortunate to enjoy a summer at Oak Ridge National Laboratory,
the highlight of which was discussing parallel rendering research with
Sean Ahern, a Chromium author.  I was that kid that set up Chromium in
his dorm room just because he `thought it was cool'; years later, Sean
would treat me as an equal.  Thanks, Sean.

I am grateful to Gunther Weber and Mark Howison, who helped us identify
the research challenges and questions we wanted to address when our HPG
volume rendering paper was less than an inkling of an idea.  Gunther
and I have had many interesting AMR volume rendering conversations, as
well.

Chuck Hansen has been especially helpful in helping me to formulate
relevant ideas and present them effectively.  Thanks!

%Chuck offers help
%whenever I ask without delay or expectation of anything in return.

Chris Johnson deserves a special thanks.  He was the perfect manager
while I was at SCI: his door was always open, the research challenges
were always beyond measure, the queue of collaborators grew faster than
it shrank, direction was available but never prescribed, and a golden
road of funding grew wherever I wandered.  Perhaps the best testament
to his style is that it is only now, looking back, that I realize I was
working for him and not myself.

Special thanks are due to friends in my research group, notably
Alexander Schiewe and Andrey Krekhov, for copious supplies of
sanity-inducing beer.  Wouldn't have been the same without you
guys---thanks.

Some computations described in this work were performed using the
\href{http://enzo-project.org}{Enzo code}, which is the product of a
collaborative effort of scientists at many universities and national
laboratories.  I especially thank Matthew Turk and Sam Skillman for
their help interfacing with \texttt{yt}.  I thank Burlen Loring for
help with ParaView scripting.

This research was made possible in part by the Intel Visual
Computing Institute; the NIH/NCRR Center for Integrative Biomedical
Computing, P41-RR12553-10; Award Number R01EB007688 from the National
Institute Of Biomedical Imaging And Bioengineering; the Office of
Advanced Scientific Computing Research, Office of Science, of the
U.S. Department of Energy under Contract No. DE-AC02-05CH11231
through the Scientific Discovery through Advanced Computing (SciDAC)
program's Visualization and Analytics Center for Enabling Technologies
(VACET); by the Cluster of Excellence `Multimodal Computing and
Interaction' at Saarland University; by the Center for the Simulation
of Accidental Fires and Explosions at the University of Utah, which was
funded by the U.S. Department of Energy under Contract No. B524196,
with supporting funds provided by the University of Utah Research
fund. Resources were utilized at the Texas Advanced Computing Center
(TACC) at the University of Texas at Austin and at the National Center
for Computational Sciences at Oak Ridge National Laboratory, which is
supported by the Office of Science of the U.S. Department of Energy
under Contract No. DE-AC05-00OR22725.

The content is under sole responsibility of the author.
