We are generating more data than we could possibly analyze.


The absurd scale implies that visualization's role will be increasingly
critical.


Extreme data sizes as well as established algorithms (in SciVis)
mean there is increasing focus on constants in algorithmic scaling
equations.

The best-represented subset of scientific data is regularly gridded
data.

% * vis is important
%
% * performance is critical
%
% * prevalence of regular grids

\section{Volume visualization}

%* volume rendering
%	* definition
%	* why is it used
%	* imagevis3d (not tuvok!)

Volume visualization is a useful technique for understanding the
structure of 3D data.

Volume rendering allows us to see inside data sets.


A \emph{transfer function} gives user control over the filtering and
mapping processes of the visualization pipeline.


Volume rendering is computationally complex.

\section{Systems opportunities}

There are a number of systems-oriented challenges in the ameloriation of
algorithmic constants in volume visualization.

\subsection{Hardware \& programmability}

The end of Moore's law necessitates a reorganization in software architecture
to take advantage of novel architectures.

In part due to the results of this thesis, architecting for
accelerators as opposed to CPU threads holds greater promise for
long-term performance scalability.

The exact characteristics of accelerators is a current topic of
industry competition, but the general characteristics are large numbers
of low-power cores connected to limited but high-bandwidth memory.

The programmability of future high-performance systems is a competitive
topic that is presently conflated with that of the hardware.

%	* GPUs
%	* versus CPU threads
%	* versus Phi?
%	* future architecture of supercomputers defined by current research
%	* programmability
%	* CUDA, OpenCL, OpenMP, OpenAcc

\subsection{I/O}

The storage hierarchy is the single most limiting architectural
component.

Parallel storage scalability is not a simple as adding more disks.

%* parallel io
%	* filesystems
%	* lustre
%	* MDSs, ODSs or whatever they're called
%	* DDoS metadata
%	* false sharing

There are no imminent advances on the horizon for the IO problems
plaguing modern visualization and analysis software.

\subsection{\textit{In situ} visualization}

In situ visualization addresses the `too big to read' problem in
visualization.

In situ visualization exposes difficulties in coupling visualization
and simulation codes.

%* in situ visualization
%	* solves
%		* data too large to be read
%		* end-to-end 'time-to-insight' performance
%	* problems
%		* how metadata is transferred
%		* vis cannot slow down sim (much)
%		* data access from sim -> vis
%		* difficulty in coupling sim+vis
%		* how often do we update vis
%		* \emph{when} do we update vis
