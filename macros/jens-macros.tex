%
% jens-macros.tex
%

%% Jens: improve typesetting quality
\usepackage{microtype}

%% Jens: Color include, needed for some of the macros below
\usepackage[usenames]{color}
%% Jens: The \anonymizeForReview macro automatically replaces text with the word
%%       "anonymized" in bold gray if a "review" documentclass is choosen
%%        otherwise it's a NOOP
%\ifreviewelse{\newcommand{\anonymizeForReview}[1]{\textcolor[rgb]{0.50,0.50,0.50}{\textbf{anonymized}}}}{\newcommand{\anonymizeForReview}[1]{#1}}
%% Jens: The \TODO macro is used to flag text that should
%%       not make it into the submitted version, it is
%%       compiled to red text and should also be easy to
%%       find by a search call in the tex file before
%%       submission.
\newcommand{\TODO}[1]{\textcolor[rgb]{1.00,0.00,0.00}{\textbf{#1}}}
\newcommand{\todo}[1]{\textcolor[rgb]{1.00,0.00,0.00}{\textbf{#1}}}
%\newcommand{\TODO}[1]{}
%% Jens: Helper for the \CE macro below
\makeatletter
\def\ifEmpty#1{\def\@temp{#1}\ifx\@temp\@empty}
\makeatother
%% Jens: The the \isDraft macro to true to replace all images by
%%       (correctly sized) boxes for faster preview
\newcommand{\isDraft}{false}
%% Jens: for align environment
\usepackage{amsmath,amsfonts,amssymb}
%% for using urls
\definecolor{darkblue}{rgb}{0,0,0.75}
%% Jens: get rid of the ifpdf clash (needed for the hyperrefs below)
\makeatletter
\let\saved@ifpdf\ifpdf
\let\ifpdf\@undefined
\usepackage{ifpdf}
%\let\ifpdf\saved@ifpdf
%\makeatother
%% Jens: turn refs into links and give them a blue color (remove for print version)
%%\usepackage[colorlinks=true,linkcolor=darkblue,citecolor=darkblue,urlcolor=darkblue]{hyperref}
%% Jens: Define a new 'tinyurl' style for the package that will use a smaller font.
%%       this can be activated in the references by inserting: \urlstyle{tinyurl}
%\makeatletter

\usepackage{graphics,graphicx}

% Math Commands
\newcommand{\mat}[1] {\boldsymbol{#1}} %{#1}
\newcommand{\vect}[1]{\boldsymbol{#1}}
\newcommand{\uvect}[1]{\boldsymbol{\hat{#1}}}
\newcommand{\norm}[1]{\lVert#1\rVert}
\newcommand{\abs}[1]{\lvert#1\rvert}
\newcommand{\transp}[1]{{#1}^\top}
\newcommand{\invtransp}[1]{{#1}^{-\top}}
\newcommand{\inv}[1]{{#1}^{-1}}
\newcommand{\scprod}[2]{#1\cdot#2}
\newcommand{\inprod}[2]{\left<#1,#2\right>}
\newcommand{\real}{\mathbb{R}}
\newcommand{\rthree}{\reel^3}
\newcommand{\cmplx}{\mathbb{C}}
\newcommand{\ints}{\mathbb{Z}}
\newcommand{\conj}[1]{\overline{#1}}

\newcommand{\SC}[1]{Sec.~\ref{#1}}
\newcommand{\SCp}[1]{Section~\ref{#1} on page~\pageref{#1}}
\newcommand{\EQWB}[1]{(Eq.~\ref{#1})}
\newcommand{\EQ}[1]{Eq.~\ref{#1}}
\newcommand{\EQp}[1]{Equation~\ref{#1} on page~\pageref{#1}}
\newcommand{\FG}[1]{Fig.~\ref{#1}}
\newcommand{\FGp}[1]{Figure~\ref{#1} on page~\pageref{#1}}
\newcommand{\TA}[1]{Table~\ref{#1}}
\newcommand{\TAp}[1]{Table~\ref{#1} on page~\pageref{#1}}
\newcommand{\AL}[1]{Algorithm~\ref{#1}}
\newcommand{\ALp}[1]{Algorithm~\ref{#1} on page~\pageref{#1}}

\DeclareMathOperator{\sinc}{sinc}
\DeclareMathOperator{\mmid}{mid}
\DeclareMathOperator{\sincBCC}{sincBCC}
\DeclareMathOperator{\ramp}{\mathcal{R}}
\DeclareMathOperator{\boxx}{\mathcal{B}}
\DeclareMathOperator{\step}{\mathcal{H}} %{Heaviside}
\DeclareMathOperator{\tesseract}{\mathcal{T}}
\DeclareMathOperator{\hatfcn}{\Lambda}
\DeclareMathOperator{\grad}{\nabla}
\newcommand{\Fourier}[1]{\mathcal{F}\{#1\}}
\newcommand{\shah}{{\textstyle \amalg{\kern-4.pt\amalg}}}
\newcommand{\myx}[1]{{x}_#1}
\newcommand{\myy}[1]{{y}_#1}
\newcommand{\myz}[1]{\mathrm{z}_#1}
\newcommand{\myw}[1]{\mathrm{w}_#1}
\newcommand{\myxi}[1]{\vect{\xi}_#1^\perp}
% I really hate TeX sometimes.
\newcommand{\tjftilde}{\raise.17ex\hbox{$\scriptstyle\mathtt{\sim}$}}
